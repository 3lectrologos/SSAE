\documentclass[a4paper, 11pt, english, greek]{article}

\usepackage{babel}
\usepackage{ucs}
\usepackage[utf8x]{inputenc}

\usepackage[T1]{fontenc}
%%\usepackage{lmodern}
\renewcommand{\ttdefault}{pcr}

\usepackage{subfig}
\usepackage[pdftex]{graphicx}
%%\usepackage{cite}
\usepackage{latexsym}
\usepackage{amsmath}

\title{Σχεδίαση Συστημάτων Αυτομάτου Ελέγχου \\ \vspace{12pt}
Εργαστηριακή άσκηση \textlatin{Matlab/Simulink}}
\author{Άλκης Γκότοβος}

\usepackage{hyperref}
\usepackage[all]{hypcap}

\begin{document}

\begin{titlepage}
	\maketitle
	\thispagestyle{empty}
\end{titlepage}

\section{Στόχος}
Στόχος της άσκησης είναι η σχεδίαση τριών διαφορετικών ελεγκτών για τον έλεγχο ενός συστήματος διασυνδεδεμένων
δεξαμενών που χρησιμοποιείται σε υδροηλεκτρικό εργοστάσιο, έτσι ώστε να πληρούνται δεδομένες προδιαγραφές.

Αρχικά θα σχεδιασθεί ένας \emph{ελεγκτής ανάδρασης κατάστασης},
στη συνέχεια ένας \emph{παρατηρητής πλήρους τάξης} και
τέλος ένας \emph{παρατηρητής μειωμένης τάξης}.

\section{Σύστημα ανοιχτού βρόχου}
Σύμφωνα με την εκφώνηση το σύστημα των δεξαμενών διέπεται από τις παρακάτω διαφορικές εξισώσεις:
\begin{equation}
  \begin{split}
  	\label{eq:dif}
	q_i(t)-q(t) &= A_1 \dot{h}_1(t)\\
	q(t)-q_0(t) &= A_2 \dot{h}_2(t)\\
	h_1(t) - h_2(t) &= q(t) R_1\\
	h_2(t) &= q_0(t) R_2
  \end{split}
\end{equation}
Χρησιμοποιώντας ως μεταβλητές κατάστασης τις $x_1(t)=q_0(t)$ και $x_2(t)=h_1(t)$ και είσοδο $u(t) = q_i(t)$,
μπορούμε να εξάγουμε τις εξισώσεις που περιγράφουν το σύστημα στο χώρο κατάστασης:
\begin{equation}
  \begin{split}
  	\label{eq:ss}
  	\dot{\mathbf{x}} &= \mathbf{A}\mathbf{x} + \mathbf{B}\mathbf{u}\\
    y &= \mathbf{C}\mathbf{x}
  \end{split}
\end{equation}
όπου
\begin{equation}
  \label{eq:mat}
  \begin{split}
    \mathbf{A} &=
  	\begin{bmatrix}
      -(\frac{\displaystyle 1}{\displaystyle R_1 A_2} + \frac{\displaystyle 1}{\displaystyle R_2 A_2}) &
      \frac{\displaystyle 1}{\displaystyle R_1 R_2 A_2} \\
      \frac{\displaystyle R_2}{\displaystyle R_1 A_1} &
      -\frac{\displaystyle 1}{\displaystyle R_1 A_1}
    \end{bmatrix}\\
    \mathbf{B} &=
    \begin{bmatrix}
      0\\
      \frac{\displaystyle 1}{\displaystyle A_1}
    \end{bmatrix}\\
    \mathbf{C} &=
    \begin{bmatrix}
      1 & 0
    \end{bmatrix}
  \end{split}
\end{equation}

%%\begin{figure}[htb]
%%  \centering
%% \includegraphics[width=300px]{test1}
%%  \caption{}
%%\end{figure}

\end{document}